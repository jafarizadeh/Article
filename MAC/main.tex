\documentclass[11pt,a4paper]{article}
\usepackage[a4paper, margin=1in]{geometry}
\usepackage{graphicx}
\usepackage{setspace}
\usepackage{amsmath, amssymb} 
\usepackage{lmodern} 
\usepackage{hyperref} 
\usepackage{xcolor}
\usepackage{listings}

\lstset{
  backgroundcolor=\color{gray!30},  
  basicstyle=\small\ttfamily\color{black},  
  keywordstyle=\bfseries\color{cyan},  
  commentstyle=\itshape\color{green},  
  stringstyle=\color{orange},  
  showstringspaces=false,  
  frame=single,  
  breaklines=true  
}


\title{Media Access Control (MAC)}
\author{Mehdi JAFARIZADEH}
\date{June 29, 2024}

\begin{document}

\maketitle

\begin{abstract}
    test
\end{abstract}


\section*{Introduction to MAC Addresses}
In networking, a Media Access Control (MAC) address plays a crucial role. It enables devices to communicate smoothly on a local network. A MAC address is a unique identifier. This identifier is linked to a network interface controller (NIC) and is essential for communication within a network segment. It’s important for the proper operation of network protocols at the data link layer. This ensures that data is delivered accurately to its target.

MAC addresses are often called hardware addresses. They are 48-bit identifiers, usually shown as six pairs of hexadecimal digits like 00:1A:2B:3C:4D:5E. Each network device—whether it's a computer, router, switch, or smartphone—has its own unique MAC address. This number helps it distinguish from other devices in the same network.

The first three octets of a MAC are set by the IEEE to show the manufacturer. The last three octets are given by that manufacturer to keep things unique within their products.\footnote{\href{https://standards.ieee.org/products-programs/regauth/}{\textit{IEEE Registration Authority: MAC Address}}}


\subsection*{Importance of MAC Addresses in Networking}

MAC addresses play a key role in how networks work. They help ensure that data packets are transmitted accurately. When one device sends a data packet to another within the same local network, the MAC address is what allows that packet to arrive correctly. In switched networks, when a device sends out such a packet, the switch looks at the MAC address to find out which port to use for sending it along. This method is called MAC address learning. The switch builds a table that links each MAC address with its specific port.

Furthermore, MAC addresses are also important for network security and management. Network administrators monitor MAC addresses to manage device access. They can set rules for filtering and diagnose network issues more effectively. Moreover, these addresses are crucial for VLAN segmentation, ensuring that traffic remains properly separated and directed in various sections of the network.


\section*{Understanding MAC Addresses: Structure and Function}
To comprehend the significance of MAC addresses in networking, it’s vital to understand their structure and function. This section explores the format and makeup of MAC addresses, along with how these addresses work within a network for effective data communication.


\subsection*{Format and Composition of MAC Addresses}

A MAC address is a 48-bit identifier. It is typically shown in two formats: six groups of two hexadecimal digits separated by colons (e.g., 00:1A:2B:3C:4D:5E) or hyphens (e.g., 00-1A-2B-3C-4D-5E). Each group consists of 8 bits, which equals one byte. In total, there are 6 bytes, adding up to 48 bits.

This identifier has two main parts:

\begin{enumerate}
    
    \item \textbf{Organizationally Unique Identifier (OUI):} The first 24 bits, known as the OUI, represent the first three octets. The IEEE assigns this segment to the manufacturer of the network device. It helps indicate the source of the hardware.

    \item \textbf{Device Identifier:} The last 24 bits, or the final three octets, are uniquely assigned by the manufacturer to each device. This process ensures that every MAC address from that manufacturer is distinct.

\end{enumerate}


Together, these two parts—the OUI and device identifier—ensure that every network interface controller (NIC) worldwide has a unique MAC address. This uniqueness is foundational for reliable network communication.\footnote{\href{https://standards.ieee.org/products-programs/regauth/mac.html}{\textit{IEEE Standards Association: MAC Address Structure}}}

\subsection*{How MAC Addresses Function in a Network}

In any network, MAC addresses are for identifying devices and communication. When one wants to send information to another in the same local area network (LAN), it needs the recipient's MAC address. This ensures that the data Is correctly delivered The process is outlined as follows:

\begin{enumerate}
    
    \item \textbf{Address Resolution Protocol (ARP):} Before sending data, a device uses ARP. This protocol helps map the recipient's IP address to its MAC address. ARP sends out a request to all devices in the network, and then the device with the matching IP address replies back with its MAC address.

    \item \textbf{Data Transmission:} After learning the recipient’s MAC address, the sending device encapsulates the data packet into an Ethernet frame with that MAC address. This frame then gets transmitted via network switches.

    \item \textbf{Switch Operation:} When a switch gets a data frame, it looks at the destination MAC address. It checks its MAC address table to find out which port to use for sending it on. If it doesn't find the address in its table, it sends the frame out to every port except for the one it came in on. This way, it can still reach who it's meant for.


    \item \textbf{Learning and Forwarding:} As switches handle more data traffic, they continually build their MAC address table by recording each incoming frame’s source MAC address and linking it to the corresponding switch port. This ongoing learning process enhances network efficiency by allowing switches to send frames specifically to the correct ports, thereby reducing unnecessary network traffic.

\end{enumerate}

This entire process highlights the critical role MAC addresses play in ensuring data is delivered to its intended destination accurately, reducing collisions, and improving overall network performance.


\section*{Role of MAC Addresses in Cisco Switches}

Cisco switches are effective at managing network traffic effectively, largely thanks the way they use MAC addresses. This section discusses how these switches utilize MAC. It also highlights the MAC address table is so significant in this operation.


\subsection*{How Cisco Switches Utilize MAC Addresses}

Cisco switches depend on MAC addresses to make sure data packets arrive at the right place in a local area network (LAN). When they receive an Ethernet frame, they check both the source and destination MAC addresses. This helps the switch decide what to do next. Here’s a closer look at how Cisco switches utilize MAC addresses:

\begin{enumerate}
    \item \textbf{Frame Forwarding:} When a data frame comes in, the switch looks at the destination MAC address. If it recognizes it and has it linked to a certain port in its MAC address table, it sends the frame only to that port. This way, there's less unnecessary network traffic. It makes everything more efficient.

    \item \textbf{Learning Process:} Cisco switches also learn about the MAC addresses of devices connected to each port over time. Whenever a frame gets to a specific port, the switch records the source MAC address and links it with that port. This learning helps keep the MAC address table current. It's important for forwarding frames correctly.

    \item \textbf{Broadcast Handling:} If the destination MAC address isn’t found in the table, the switch broadcasts the frame to all ports except the one it originated from. This ensures that the frame reaches its intended device. The device will then respond, allowing the switch to learn and record its MAC address for future use.

    \item \textbf{Security Features:} Additionally, Cisco switches use MAC addresses for supporting various security measures. Features like port security allow administrators to limit which devices can connect to certain ports based on their MAC addresses. This is vital for preventing unauthorized access and improving overall network safety.

\end{enumerate}


\end{document}