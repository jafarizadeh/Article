\documentclass[11pt,a4paper]{article}
\usepackage[a4paper, margin=1in]{geometry}
\usepackage{graphicx}
\usepackage{setspace}
\usepackage{amsmath, amssymb} 
\usepackage{lmodern} 
\usepackage{hyperref} 
\usepackage{xcolor}
\usepackage{listings}

\lstset{
  backgroundcolor=\color{gray!30},  
  basicstyle=\small\ttfamily\color{black},  
  keywordstyle=\bfseries\color{cyan},  
  commentstyle=\itshape\color{green},  
  stringstyle=\color{orange},  
  showstringspaces=false,  
  frame=single,  
  breaklines=true  
}


\title{Media Access Control (MAC)}
\author{Mehdi JAFARIZADEH}
\date{June 29, 2024}

\begin{document}

\maketitle

\begin{abstract}
    test
\end{abstract}


\section*{Introduction to MAC Addresses}
In networking, a Media Access Control (MAC) address plays a crucial role. It enables devices to communicate smoothly on a local network. A MAC address is a unique identifier. This identifier is linked to a network interface controller (NIC) and is essential for communication within a network segment. It’s important for the proper operation of network protocols at the data link layer. This ensures that data is delivered accurately to its target.

MAC addresses are often called hardware addresses. They are 48-bit identifiers, usually shown as six pairs of hexadecimal digits like 00:1A:2B:3C:4D:5E. Each network device—whether it's a computer, router, switch, or smartphone—has its own unique MAC address. This number helps it distinguish from other devices in the same network.

The first three octets of a MAC are set by the IEEE to show the manufacturer. The last three octets are given by that manufacturer to keep things unique within their products.\footnote{\href{https://standards.ieee.org/products-programs/regauth/}{\textit{IEEE Registration Authority: MAC Address}}}


\subsection*{Importance of MAC Addresses in Networking}

MAC addresses play a key role in how networks work. They help ensure that data packets are transmitted accurately. When one device sends a data packet to another within the same local network, the MAC address is what allows that packet to arrive correctly. In switched networks, when a device sends out such a packet, the switch looks at the MAC address to find out which port to use for sending it along. This method is called MAC address learning. The switch builds a table that links each MAC address with its specific port.

Furthermore, MAC addresses are also important for network security and management. Network administrators monitor MAC addresses to manage device access. They can set rules for filtering and diagnose network issues more effectively. Moreover, these addresses are crucial for VLAN segmentation, ensuring that traffic remains properly separated and directed in various sections of the network.


\section*{Understanding MAC Addresses: Structure and Function}
To comprehend the significance of MAC addresses in networking, it’s vital to understand their structure and function. This section explores the format and makeup of MAC addresses, along with how these addresses work within a network for effective data communication.


\subsection*{Format and Composition of MAC Addresses}

A MAC address is a 48-bit identifier. It is typically shown in two formats: six groups of two hexadecimal digits separated by colons (e.g., 00:1A:2B:3C:4D:5E) or hyphens (e.g., 00-1A-2B-3C-4D-5E). Each group consists of 8 bits, which equals one byte. In total, there are 6 bytes, adding up to 48 bits.

This identifier has two main parts:

\begin{enumerate}
    
    \item \textbf{Organizationally Unique Identifier (OUI):} The first 24 bits, known as the OUI, represent the first three octets. The IEEE assigns this segment to the manufacturer of the network device. It helps indicate the source of the hardware.

    \item \textbf{Device Identifier:} The last 24 bits, or the final three octets, are uniquely assigned by the manufacturer to each device. This process ensures that every MAC address from that manufacturer is distinct.

\end{enumerate}


Together, these two parts—the OUI and device identifier—ensure that every network interface controller (NIC) worldwide has a unique MAC address. This uniqueness is foundational for reliable network communication.\footnote{\href{https://standards.ieee.org/products-programs/regauth/mac.html}{\textit{IEEE Standards Association: MAC Address Structure}}}

\subsection*{How MAC Addresses Function in a Network}

In any network, MAC addresses are for identifying devices and communication. When one wants to send information to another in the same local area network (LAN), it needs the recipient's MAC address. This ensures that the data Is correctly delivered The process is outlined as follows:

\begin{enumerate}
    
    \item \textbf{Address Resolution Protocol (ARP):} Before sending data, a device uses ARP. This protocol helps map the recipient's IP address to its MAC address. ARP sends out a request to all devices in the network, and then the device with the matching IP address replies back with its MAC address.

    \item \textbf{Data Transmission:} After learning the recipient’s MAC address, the sending device encapsulates the data packet into an Ethernet frame with that MAC address. This frame then gets transmitted via network switches.

    \item \textbf{Switch Operation:} When a switch gets a data frame, it looks at the destination MAC address. It checks its MAC address table to find out which port to use for sending it on. If it doesn't find the address in its table, it sends the frame out to every port except for the one it came in on. This way, it can still reach who it's meant for.


    \item \textbf{Learning and Forwarding:} As switches handle more data traffic, they continually build their MAC address table by recording each incoming frame’s source MAC address and linking it to the corresponding switch port. This ongoing learning process enhances network efficiency by allowing switches to send frames specifically to the correct ports, thereby reducing unnecessary network traffic.

\end{enumerate}

This entire process highlights the critical role MAC addresses play in ensuring data is delivered to its intended destination accurately, reducing collisions, and improving overall network performance.


\section*{Role of MAC Addresses in Cisco Switches}

Cisco switches are effective at managing network traffic effectively, largely thanks the way they use MAC addresses. This section discusses how these switches utilize MAC. It also highlights the MAC address table is so significant in this operation.


\subsection*{How Cisco Switches Utilize MAC Addresses}

Cisco switches depend on MAC addresses to make sure data packets arrive at the right place in a local area network (LAN). When they receive an Ethernet frame, they check both the source and destination MAC addresses. This helps the switch decide what to do next. Here’s a closer look at how Cisco switches utilize MAC addresses:

\begin{enumerate}
    \item \textbf{Frame Forwarding:} When a data frame comes in, the switch looks at the destination MAC address. If it recognizes it and has it linked to a certain port in its MAC address table, it sends the frame only to that port. This way, there's less unnecessary network traffic. It makes everything more efficient.

    \item \textbf{Learning Process:} Cisco switches also learn about the MAC addresses of devices connected to each port over time. Whenever a frame gets to a specific port, the switch records the source MAC address and links it with that port. This learning helps keep the MAC address table current. It's important for forwarding frames correctly.

    \item \textbf{Broadcast Handling:} If the destination MAC address isn’t found in the table, the switch broadcasts the frame to all ports except the one it originated from. This ensures that the frame reaches its intended device. The device will then respond, allowing the switch to learn and record its MAC address for future use.

    \item \textbf{Security Features:} Additionally, Cisco switches use MAC addresses for supporting various security measures. Features like port security allow administrators to limit which devices can connect to certain ports based on their MAC addresses. This is vital for preventing unauthorized access and improving overall network safety.

\end{enumerate}

\subsection*{MAC Address Table: A Vital Element}

The MAC address table, often referred to as Content Addressable Memory (CAM) table in Cisco switches, plays an essential role in network operations. Within this table lies the crucial mappings of MAC addresses to ports, which enables efficient packet forwarding and minimizes congestion.

\begin{enumerate}
    \item \textbf{Table Structure:} Each entry in the MAC address table links a MAC address with a specific port number on the switch. Typically, each entry includes the MAC address, its associated port number, and VLAN ID if VLANs are applied.

    \item \textbf{Dynamic Learning:} As mentioned earlier, Cisco switches dynamically build this table through their learning process. By observing frames traversing the network, they continuously update the table to accurately reflect changes in the network topology, such as when devices are added or moved.

    \item \textbf{Aging Process:} To maintain efficiency, the MAC address table undergoes an aging process. Entries are removed after being inactive for a specified period. This process prevents outdated information from cluttering the table, ensuring that only relevant records are retained.

    \item \textbf{Troubleshooting and Management:} Network admins can view and manage the MAC address table conveniently using Cisco's command-line interface (CLI). Commands such as \lstinline{show mac address-table} assist administrators in confirming which MAC addresses pair with which ports, Assisting with for troubleshooting and management tasks.

\end{enumerate}

The structure and functionality of this table are crucial for guaranteeing that Cisco switches operate efficiently and securely—acting as a foundation for effective communication and management within networks.



\section*{MAC Address Table Operations in Cisco Switches}

The MAC address table in Cisco switches is integral to for the efficient and secure operations of networks. This section explores how Cisco switches create the MAC table, make forwarding and filtering decisions, and manage the retention of MAC addresses in the table.


\subsection*{Learning Process: Building the MAC Table}

Cisco switches build and maintain the MAC address table through a dynamic learning process. This method is crucial to ensuring frames are directed to the correct destinations within the network. Here’s how this process works:

\begin{enumerate}
    
    \item \textbf{Frame Reception:} When an Ethernet frame comes into a Cisco switch on a specific port, it examines the source MAC address found inside that frame.

    \item \textbf{Updating the Table:} Next, the switch checks its MAC address table to see if this source MAC address is already linked to any port. If it isn’t in the table, the switch adds a new entry. This entry connects that MAC address to the port it was received on. This mapping enables the switch to forward future frames intended for that MAC address to the correct port.

    \item \textbf{Continuous Learning:} As more frames are received, the switch continuously updates its MAC address table, adding new entries or modifying existing ones if devices move to different ports. This ongoing learning process keeps the table accurate and reflective of the current network topology.\footnote{\href{https://learningnetwork.cisco.com/s/article/mac-address-learing-pdf}{\textit{Cisco Systems: MAC Address Learning Process}}}

\end{enumerate}


\subsection*{Forwarding and Filtering Decisions}

Once the table is populated, Cisco switches utilize this table to make forwarding and filtering choices, quite crucial for managing network traffic well.
\begin{enumerate}
    
    \item \textbf{Forwarding Frames:} When they receive a frame, switches check its destination MAC address in their table. If there's a match, they forward it only to the port with that MAC address. This direct forwarding cuts down unnecessary network traffic and enhances efficiency.

    \item \textbf{Filtering Frames:} If the destination MAC address is not found in the table, the switch must decide how to handle the frame. Typically, the switch floods the frame to all ports except the one it was received on, ensuring it reaches its destination. Additionally, Cisco switches can be configured with specific rules to drop frames based on their MAC addresses, thereby enhancing both security and performance.

    \item \textbf{Handling Broadcast and Multicast Traffic:} For broadcast and multicast traffic—meant for several devices—there's a different approach. Cisco switches automatically send broadcast frames to all ports. Meanwhile, multicast frames are sent based on how their group is set up and what’s in the MAC address table.\footnote{\href{https://www.cisco.com/c/en/us/support/docs/routers/12000-series-routers/47321-ciscoef.html}{\textit{Forwarding and Filtering Decisions}}}

\end{enumerate}


\subsection*{Aging Process of MAC Addresses}

To maintain the MAC address table from being too big or old-fashioned, Cisco switches use an aging method for these entries.

\begin{enumerate}

    \item \textbf{Aging Timer:} Every entry has an aging timer attached. Cisco switches usually set this timer at 300 seconds (or 5 minutes). If a MAC address isn’t seen during this period, it's considered stale and removed from the table.

\item \textbf{Refreshing Entries:} If a frame with a source MAC address already in the table is received, the switch resets the aging timer for that entry. This process ensures that active devices remain in the MAC address table, while inactive or disconnected devices are eventually removed.

\item \textbf{Impact of Aging:} The aging process helps maintain optimal switch performance by controlling the size of the MAC address table. It ensures that the table accurately reflects current network conditions, allowing the switch to respond appropriately when devices move or are removed.

\item \textbf{Configuring Aging Time:} Network administrators can adjust the aging time based on specific network requirements. Shorter aging times may be more appropriate for dynamic environments, while longer times might be better suited for more stable networks. Proper configuration of the aging timer is crucial for maintaining an efficient and responsive network.\footnote{\href{https://community.cisco.com/t5/networking-knowledge-base/network-switching-operation/ta-p/4193160}{\textit{MAC Address Table Aging Process}}}
\end{enumerate}

\section*{Configuring MAC Addresses on Cisco Switches}

Cisco switches allow administrators to manage MAC addresses efficiently. They can configure both and dynamic entries based on the needs of the network. In this section, we will explore the difference between static and dynamic MAC address entries, provide a guide for setting up static MAC addresses, and detail how to confirm these settings.

\end{document}