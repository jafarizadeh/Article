\documentclass[11pt,a4paper]{article}
\usepackage[a4paper, margin=1in]{geometry}
\usepackage{graphicx}
\usepackage{setspace}
\usepackage{amsmath, amssymb} 
\usepackage{lmodern} 
\usepackage{hyperref} 
\usepackage{xcolor}
\usepackage{listings}

\lstset{
  backgroundcolor=\color{gray!30},  
  basicstyle=\small\ttfamily\color{black},  
  keywordstyle=\bfseries\color{cyan},  
  commentstyle=\itshape\color{green},  
  stringstyle=\color{orange},  
  showstringspaces=false,  
  frame=single,  
  breaklines=true  
}


\title{Media Access Control (MAC)}
\author{Mehdi JAFARIZADEH}
\date{June 29, 2024}

\begin{document}

\maketitle

\begin{abstract}
    test
\end{abstract}


\section*{Introduction to MAC Addresses}
In networking, a Media Access Control (MAC) address plays a crucial role. It enables devices to communicate smoothly on a local network. A MAC address is a unique identifier. This identifier is linked to a network interface controller (NIC) and is essential for communication within a network segment. It’s important for the proper operation of network protocols at the data link layer. This ensures that data is delivered accurately to its target.

MAC addresses are often called hardware addresses. They are 48-bit identifiers, usually shown as six pairs of hexadecimal digits like 00:1A:2B:3C:4D:5E. Each network device—whether it's a computer, router, switch, or smartphone—has its own unique MAC address. This number helps it distinguish from other devices in the same network.

The first three octets of a MAC are set by the IEEE to show the manufacturer. The last three octets are given by that manufacturer to keep things unique within their products.\footnote{\href{https://standards.ieee.org/products-programs/regauth/}{\textit{IEEE Registration Authority: MAC Address}}}


\subsection*{Importance of MAC Addresses in Networking}

MAC addresses play a key role in how networks work. They help ensure that data packets are transmitted accurately. When one device sends a data packet to another within the same local network, the MAC address is what allows that packet to arrive correctly. In switched networks, when a device sends out such a packet, the switch looks at the MAC address to find out which port to use for sending it along. This method is called MAC address learning. The switch builds a table that links each MAC address with its specific port.

Furthermore, MAC addresses are also important for network security and management. Network administrators monitor MAC addresses to manage device access. They can set rules for filtering and diagnose network issues more effectively. Moreover, these addresses are crucial for VLAN segmentation, ensuring that traffic remains properly separated and directed in various sections of the network.


\section*{Understanding MAC Addresses: Structure and Function}
To comprehend the significance of MAC addresses in networking, it’s vital to understand their structure and function. This section explores the format and makeup of MAC addresses, along with how these addresses work within a network for effective data communication.


\subsection*{Format and Composition of MAC Addresses}

A MAC address is a 48-bit identifier. It is typically shown in two formats: six groups of two hexadecimal digits separated by colons (e.g., 00:1A:2B:3C:4D:5E) or hyphens (e.g., 00-1A-2B-3C-4D-5E). Each group consists of 8 bits, which equals one byte. In total, there are 6 bytes, adding up to 48 bits.

This identifier has two main parts:


\begin{enumerate}
    
    \item \textbf{Organizationally Unique Identifier (OUI):} The first 24 bits, known as the OUI, represent the first three octets. The IEEE assigns this segment to the manufacturer of the network device. It helps indicate the source of the hardware.

    \item \textbf{Device Identifier:} The last 24 bits, or the final three octets, are uniquely assigned by the manufacturer to each device. This process ensures that every MAC address from that manufacturer is distinct.

\end{enumerate}


\end{document}